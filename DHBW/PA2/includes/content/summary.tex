\chapter{Summary and Outlook}
\label{summary}
 
\section{Summary}

By way of this project paper we have assessed and described the state of the art for current technologies in the areas of 
containerized software, container orchestration and how those tie in with Software defined infrastructure.
We have done the same for the state of the art in the solution of complex problems found in the field of and  solved with \ac{HPC},
namely the two larger classes of problems, \ac{LCP} and \ac{TCP} problems, and what strategies are usually empoyed to solve those problems.

We have then addressed the problems identified by the problem statement with the statement of an initial goal, 
and structured an intervention to solve those problems by way of the creation of a prototype.
Before the iterative process of prototyping could begin we had to make a non-trivial decision on the choice of a container orchestrator/ workflow manager.
For which we defined, discussed and weighted the selection criteria, and then evaluated the available options against those criteria and 
We found our best fit through the employment of the \ac{SMART-ER} method, which was the \ac{k8s}-based orchestrator named Pachyderm.

The iterative process of prototyping was split into three main areas of focus, the infrastructure, the solution to \acp{TCP} and the integration of a complete 
\ac{CI/CD} pipeline. Each of these areas was given at least two iterations and the results of each iteration were documented, evaluated and recomendations for
future iterations were made.

The final results of the prototype were then evaluated against the initial goal and the problems identified in the problem statement,
and the resulting artifact was found to be a valid form of intervention to solve the problems identified in the problem statement,
while still being limited in scope and applicability, as was to be expected from a non production prototype. 


\section{Outlook}

% The prototype created in this project paper is a valid intervention to solve the problems identified in the problem statement,
% but still proves to have great potential for future work.
% As the time was limited and mostly spend on the furthering of the project, many avenues of high interest are still open for exploration.

% The two main areas of focus for future work are 
% the optimization of the communication stack on every level of abstraction,
% the expansion of usability functions relating to the Catalogue as these will bring the most immediate value to potential users

