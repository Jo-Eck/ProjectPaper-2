\chapter{Methodology}
\label{methodology}

\section{Prototyping}

Needs to have a methodology from the Spectrum of Methodologies for Business information systems \footcite{wildeMethodenspektrumWirtschaftsinformatikUeberblick}


Argumentation why this project is centrally a Prototyping project:

 - The research questions are directly inspired bt the needs of the customer
 - The limitations and the scope are both defined by the available resources of the business unit as well as the time constraints of the project
         and the available know-how
 -  test

 !TODO!



\newpage

\section{Decision Making}

As previously described, the methodology of Prototyping benefits from a very tight loop of iterations between the different phases of the project.
While this is highly effective in producing a good end result, it can also take many iterations and a lot of experimentation until an adequate tool or solution has been found.
As this project has to be completed within a limited time frame, it is important to invest this time wisely and to make good decisions given the available information early on in the project, \
this is especially true if there is to little time to explore.

To ensure that the decisions that are made are the best possible decisions given the available information, it is important to have a standardized, repeatable and transparent process for making rational decisions.
Over the years many different frameworks for making good decisions based on available information haven been developed.

\subsection{Weighted Sum Model}
According to Evangelos Triantaphyllou the \ac{WSM} is the most used method of decision making in practice \footcite[p. 1]{triantaphyllouIntroductionMultiCriteriaDecision2000}.

%% \[ A^{WSM-score} = \max_i \sum_{j=1}^{n} a_{ij}w_j \quad \text{for} \quad i = 1, 2, 3, ..., m \]
!TODO!

\subsection{Simple multi-attribute rating technique}

\ac{SMART} is a method grounded in Multi-attribute Utility Theory. Its simplicity lies in its approach to decision-making which involves:

1. Identifying the decision criteria relevant to the problem.
2. Assigning weights to these criteria to represent their relative importance.
3. Scoring the alternatives on each criterion.
4. Calculating a composite score for each alternative by combining the criteria scores and weights.

It essentially prioritizes options based on their weighted scores, allowing decision-makers to choose options that align most closely with their objectives. 
This is how the SMART method can be represented mathematically:

\[ x_j = \frac{\sum_{i=1}^{m} w_i a_{ij}}{\sum_{i=1}^{m} w_i}, \quad j = 1, \ldots, n. \]

The formula denotes how each alternative is scored (represented by \(x_j\)) based on its attributes (\(a_{ij}\)) and the importance weights (\(w_i\)) of these attributes\footcite[p. 6]{fulopIntroductionDecisionMaking2005}.

!TODO!