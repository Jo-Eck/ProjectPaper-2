\chapter{Methodology}
\label{methodology}

\section{Prototyping}

Needs to have a methodology from the Spectrum of Methodologies for Business information systems \footcite{wildeMethodenspektrumWirtschaftsinformatikUeberblick}


Argumentation why this project is centrally a Prototyping project:

 - The research questions are directly inspired bt the needs of the customer
 - The limitations and the scope are both defined by the available resources of the business unit as well as the time constraints of the project
         and the available know-how
 -  test

 !TODO!

\newpage


\section{Decision Making}
\label{decision_making}

As previously described, the methodology of Prototyping benefits from a very tight loop of iterations between the different phases of the project.
While this is highly effective in producing a good end result, it can also take many iterations and a lot of experimentation until an adequate tool or solution has been found.
Given the constraints of a limited time frame for this project, it becomes crucial to use this time judiciously. Rapid and precise decision-making becomes imperative, especially when there's insufficient time for exploration.

To ensure that the decisions made are the most optimal within the constraints of the available information, adopting a systematic, replicable, and transparent decision-making process becomes essential. Over the years, various frameworks have been crafted to guide decision-making, particularly when information is complex and multi-dimensional.

\subsection{Weighted Sum Model}
Evangelos Triantaphyllou suggests that the \ac{WSM} holds a prime position in real-world decision-making applications \footcite[p. 1]{triantaphyllouIntroductionMultiCriteriaDecision2000}. The WSM method, by design, mandates the assignment of specific weights to each criterion based on its relevance. Subsequent to this, every alternative is evaluated based on these weighted criteria, resulting in a cumulative score. The alternative brandishing the highest score is naturally favored as the optimal choice.

This method, despite its simplicity and direct approach, isn't without limitations. One notable drawback is its dependence on dimensionless scales. For the weights to properly reflect the criteria's importance, the scores need to be on a common, dimensionless scale, a detail not always feasible or convenient in practice.

\subsection{Simple Multi-Attribute Rating Technique}

In contrast to the Weighted Sum Model (WSM), which predominantly utilizes a
direct mathematical approach to rank alternatives based on their weighted sum
scores, the \ac{SMART} methodology offers a more comprehensive approach to
multi-criteria decision-making. While WSM is primarily concerned with simple
weighted arithmetic sums, the SMART method dives deeper, ensuring that diverse
performance values—both quantitative and qualitative—are harmonized and placed
on a common scale.

The SMART method, grounded in Multi-attribute Utility Theory (MAUT), provides a
structured framework that encompasses more than just the weighting of criteria.
It involves:

\begin{enumerate} \item Discernment of vital criteria pertinent to the decision
in focus. \item Weight allocation to each criterion in accordance to its
significance. \item Evaluation of each potential alternative against the
identified criteria, culminating in a score. \item Aggregation of these
individual scores via their associated weights, yielding a total score for every
alternative. \end{enumerate}

By adhering to the SMART framework, alternatives can be sequenced based on
their aggregated weighted scores. This systematic approach equips
decision-makers to choose solutions that align closely with their objectives.
The computational formula integral to the SMART method is:

\[ x_j = \frac{\sum_{i=1}^{m} w_i a_{ij}}{\sum_{i=1}^{m} w_i}, \quad j = 1, \ldots, n. \]

Where:
\begin{itemize}
    \item \( x_j \) epitomizes the aggregate score of an alternative.
    \item \( a_{ij} \) signifies the score awarded to an alternative concerning a designated criterion.
    \item \( w_i \) represents the weight ascribed to that specific criterion.
\end{itemize}

This method's emphasis on utility functions ensures a more nuanced and adaptable
approach to decision-making compared to models like WSM, making it suitable for
complex scenarios where criteria and alternatives are diverse in nature \footcite[p. 6]{fulopIntroductionDecisionMaking2005}.