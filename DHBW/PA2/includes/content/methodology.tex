\chapter{Methodology}
\label{methodology}

\section{Prototyping}

\textcolor{red}{TODO}

Needs to have a methodology from the Spectrum of Methodologies for Business information systems \footcite{wildeMethodenspektrumWirtschaftsinformatikUeberblick}


Argumentation why this project is centrally a Prototyping project:

\begin{itemize}
    \item  The research questions are directly inspired bt the needs of the customer
    \item   The limitations and the scope are both defined by the available resources of the business unit as well as the time constraints of the project and the available know-how
    \item Based \footcite[p. 91]{buddeWhatPrototyping1992} can be classified as a presentation prototype in which we do a  does a vertical integration of many different sysytems, according to budde this can be described as a vertical interface, as it reaches through the entire stack of technological abstraction \footcite[p. 94]{buddeWhatPrototyping1992}    \
    \item to create this prototype we will be using  Which  will be using spiral modle \footcite{boehmSpiralModelSoftware1988} 
\end{itemize} 

\newpage


\section{Decision Making}
\label{decision_making}

As previously described, the methodology of Prototyping benefits from a very tight loop of iterations between the different phases of the project.
While this is highly effective in producing a good end result, it can also take many iterations and a lot of experimentation until an adequate tool or solution has been found.
Given the constraints of a limited time frame for this project, it becomes crucial to use this time as efficiently as possible.
Sometimes, when the time does not permit a thorough exploration of 

To ensure that the decisions made are the most optimal within the constraints of the available information, adopting a systematic, replicable, and transparent decision-making process becomes essential. Over the years, various frameworks have been crafted to guide decision-making, 
particularly when information is complex and multi-dimensional.

\subsection{Weighted Sum Model}
Evangelos Triantaphyllou suggests that the \ac{WSM} is in practice the most used and most relevant decision-making framework \footcite[p. 1]{triantaphyllouIntroductionMultiCriteriaDecision2000}.
The \ac{WSM} method, by design, mandates the assignment of specific weights to each criterion based on its relevance. Subsequent to this, every alternative is evaluated based on these weighted criteria, resulting in a cumulative score.
The alternative with the highest score is therefore the optimal choice.


\begin{figure}[h]
    \centering
    \Large
    \[ A_{i}^{WSM-score} = \sum_{j=1}^{n} w_{j} a_{ij} \quad \ for \ i = 1, 2, 3, \dots, m. \]
    \caption{Formula for calculating the \ac*{WSM} score\protect\footnotemark} 
\end{figure}
\footnotetext{\cite{WeightedSumModel2022}}

Where:
\begin{itemize}
    \item \(w_{j}\): This represents the weight assigned to the \(j\)-th criterion. Weights are determined by the decision-makers based on the relative importance of each criterion. They should be normalized (i.e., the sum of all weights should be 1 or 100%) to maintain a consistent scale.
    
    \item \(a_{ij}\): This represents the score or rating of the \(i\)-th alternative concerning the \(j\)-th criterion. This score is an assessment of how well the alternative meets or satisfies the specific criterion. 
    \end{itemize}

This method, despite its simplicity and direct approach, isn't without limitations.
One notable drawback is its dependence on dimensionless scales.
For the weights to properly reflect the criteria's importance, the scores need to be on a common, dimensionless scale, a detail not always feasible or convenient in practice.

\subsection{Simple Multi-Attribute Rating Technique}

In contrast to the \ac{WSM} , which predominantly utilizes a
direct mathematical approach to rank alternatives based on their weighted sum
scores, the \ac{SMART} methodology offers a more comprehensive approach to
multi-criteria decision-making. While \ac{WSM} is primarily concerned with simple
weighted arithmetic sums, the \ac{SMART} method dives deeper, ensuring that diverse
performance values—both quantitative and qualitative are harmonized and placed
on a common scale.

The \ac{SMART} method, grounded in \ac{MAUT}, provides a
structured framework that encompasses more than just the weighting of criteria.
It involves:

\begin{enumerate} \item Discernment of vital criteria pertinent to the decision
in focus. \item Weight allocation to each criterion in accordance to its
significance. \item Evaluation of each potential alternative against the
identified criteria, culminating in a score. \item Aggregation of these
individual scores via their associated weights, yielding a total score for every
alternative. \end{enumerate}

By adhering to the \ac{SMART} framework, alternatives can be sequenced based on
their aggregated weighted scores. This systematic approach equips
decision-makers to choose solutions that align closely with their objectives.
The computational formula integral to the \ac{SMART} method is:


\begin{figure}[h]
    \centering
    \Large
    \[ x_j = \frac{\sum_{i=1}^{m} w_i a_{ij}}{\sum_{i=1}^{m} w_i}, \quad j = 1, \ldots, n. \]
    \caption{Formula for calculating the \ac*{SMART} score\protect\footnotemark} 
\end{figure}

\footnotetext{Taken from \cite[p. 6]{fulopIntroductionDecisionMaking2005}}

Where:
\begin{itemize}
    \item \( x_j \) Is the overall utility socre for alternative \( j \). The higher the score, the better the alternative, in comparison to the other alternatives.
    \item \( a_{ij} \)  Is the utility score for alternative \( j \) for the criterion \( i \).
    \item \( w_i \) Is the weight of criterion \( i \).
\end{itemize}

This method's emphasis on utility functions ensures a more nuanced and adaptable
approach to decision-making compared to models like WSM, making it suitable for
complex scenarios where criteria and alternatives are diverse in nature \footcite[p. 6]{fulopIntroductionDecisionMaking2005}.

\subsection{SMART Exploiting Ranks}
\label{smart_er}

The \ac{SMART-ER} method is a variant of the \ac{SMART} method that attempts to alleviate the largest issue \footcitetext[p. 26]{barfodMulticriteriaDecisionAnalysis2014} of the original \ac{SMART} method, 
namely the problem of a somewhat arbitrary ranking of the options if no numerical values can be derived.

This method addresses the issue by letting the decision maker simply ranking the different criteria in relation to each other and then normalizing the weights\footcite[p. 296]{robertsWeightApproximationsMultiattribute2002a}.
They propose the different weighting curves.

\begin{figure}[h]
    \centering
    \[ w_{i}(ROC) = \frac{1}{n} \sum_{j=1}^{n} \frac{1}{j}, \quad i = 1, \ldots, n. \]
    \caption{Formula for the \ac{ROC} weights}
\end{figure}

The \ac{ROC} takes the centroid of the rank order and uses the reciprocal of the rank as the weight.

\begin{figure}[h]
    \centering
    \[ w_{i}(\ RS\ ) = \frac{n + 1 - i}{n(n + 1)/2}, \quad i = 1, \ldots, n. \]
    \caption{Formula for the \ac{RS} weights}
\end{figure}

The \ac{RS} uses linear curve where weights are normalized by dividing them by the sum of all weights.

\begin{figure}[h]
    \centering
    \[ w_i(RR) = \left( \frac{\frac{1}{i}}{\sum_{j=1}^{n} \frac{1}{j}} \right) , \ rank\ i = 1, \ldots, n, \ option\ j = 1, \ldots, n \]
    \caption{Formula for the \ac{RR} weights}
\end{figure}

The \ac{RR} emphasizes the most important criteria by using the reciprocal of the rank as the weight, then normalizing each weight by the sum of all reciprocals.