\chapter{Introduction}
\label{Introduction}

In this section, the underlying motivation of this project is explained.
Furthermore, the problems which will be addressed by this project are described,
which serve as the basis for the research questions which will guide this project and ultimately
result in solutions and further questions which are listed in the contributions section and discussed in the conclusion.

\section{Motivation}
The proliferation of "Big Data" has led to the need to compute, analyse, and visualize ever-increasing amounts of datasets,
which themselves are getting more and more complex, has led to an ever increasing demand for more efficient and quicker ways to process data.

Both the \ac{HPC} and the \ac{CC} community have been working on solutions to distribute and parallelize computations for decades, 
both with their own approaches and solutions to their respective problems.

While the \ac{HPC} community has been putting a lot of effort into developing new and extremely efficient ways to parallelize computations,
the \ac{CC} community has been focusing on improving the flexibility, scalability and resilience of their solutions as well as improving the ease of use for their developers and users. 

Both used to be very distinct and separate communities due to their very different usecases, while the \ac{HPC} community was mostly concerned with scientific computing and simulations of physical phenomena,
the \ac{CC} community is mostly concerned with providing a reliable and easily up and down scalable infrastructure for the industry and businesses.

Now with the advent of \ac{ML} and \ac{AI} the two communities are starting to converge, 
as the \ac{ML} and \ac{AI} community is adopting the tools and techniques of both communities to solve their problems as they see fit.

But this convergence of the two is not without its problems, being developed in two coexisting and separate communities, the tools and techniques of both communities are not always compatible with each other,
the goal of this project is to find a way to bridge this gap and to find a way to combine the best of both worlds.

\section{Problem Statement}


\section{Research Questions}

\begin{itemize}
    \item \textbf{RQ1:} \textit{platzhalter}
    \item \textbf{RQ2:} \textit{platzhalter}
\end{itemize}

\section{Contributions}

\begin{itemize}
    \item platzhalter   
    \item platzhalter   
\end{itemize}