\chapter*{Appendix}
\addcontentsline{toc}{chapter}{Appendix}
\section*{Appendix Index}
\vspace{-8em}

% Adjust indentations before \listofanhang
\abstaendeanhangverzeichnis

\listofanhang
\clearpage
\spezialkopfzeile{Appendix} 

\lstset{
  language=TeX, % define keywords to highlight
  morekeywords={Appendix, anhangteil, anhan},
  breaklines=true,  % Enable line wrapping within the listing
  breakatwhitespace=true, % Breaks only at white space
  postbreak=\mbox{\textcolor{red}{$\hookrightarrow$}\space}, % Optional - for having a symbol at the point where text breaks
  basicstyle=\footnotesize\ttfamily, % for setting the font size/style for the code
  tabsize=2, % sets default tabsize
}


%% ================================== Discussion =======================================

\anhang{Discussion of Tool Evaluation and Weighing}

\anhangteil{Extensibility}

\anhangteil{Community, Support \& Docs}

This section describes how much external support is given, as this support can come in many different shapes and forms we will be especially focusing on three different aspects and combine them into one score.
First, we are looking at the community size, as a large community is a good indicator for the maturity of the project and the amount of support available.
As a proxy for this, we will look at two central metrics, the amount of stars on Github and the amount of questions on Stackoverflow.

\begin{table}[h!]
    \centering 
    \caption{Comparison of project popularity}
    \label{tab:table1} 
    \begin{tabular}{lcc} 
      \textbf{Project} & \textbf{GitHub Stars} & \textbf{Stack Overflow Questions}\\ 
      Pachyderm  & 6,000   & 6\\
      Argo       & 14,500  & 136\\
      Clasp      & 0       & 0\\
      Snaplogic  & 0       & 57\\
      Airflow    & 32,200  & 10,218\\
      Kubeflow   & 13,100  & 434\\
      Knative    & 4,100   & 204\\
      Luigi      & 16,900  & 346\\
      CWL        & 1,400   & 6\\
   \end{tabular}


\end{table}

\begin{figure}[htb]
    \centering
    \includegraphics[width=12cm]{graphics/Stars_stackoverflow_comparison.png}
    \caption[Stars and Stackoverflow Questions Comparison]{Stars and Stackoverflow Questions Comparison}
    \label{abb:stars_stackoverflow_comparison}
\end{figure}

To quantify the support and community engagement around these projects, we introduce a composite score that normalizes and combines the GitHub stars and Stack Overflow questions metrics. This score is calculated using the following methodology:

Each project is represented as a point \(P_i = (x_i, y_i)\) in a two-dimensional space, with \(x_i\) and \(y_i\) being the number of GitHub stars and Stack Overflow questions, respectively, for the \(i\)-th project. The composite score \(S_i\) for each project is computed by normalizing these values to a 0-10 scale and then averaging them:

\[
S_i = \frac{1}{2} \left( \frac{x_i - \min(x)}{\max(x) - \min(x)} \times 10 + \frac{y_i - \min(y)}{\max(y) - \min(y)} \times 10 \right) \quad \ for\ \quad i = 1, 2, ..., n
\]

Here, \(\min(x)\), \(\max(x)\), \(\min(y)\), and \(\max(y)\) represent the minimum and maximum values of GitHub stars and Stack Overflow questions across all projects, respectively. The final scores \(S_i\) provide a balanced metric for comparing the projects' popularity and community engagement, taking both aspects into account on the same scale.




\anhangteil{License}

As discussed in section \ref{crit:license} the tools in consideration should not be to restrictive.
To evaluate the criteria we will employ a 4 bucket system, where tools with a permmisive license get a value of 7.5 , tools where the licensing is ideal get a 10, and tools which have a not acceptable license get weighted with 0.

\begin{itemize}
    \item \textbf{Pachyderm} 
    The licensing model of Pachyderm follows a model which has similarities with the "Open Core model" \footcite{PahcydermPricing2022}.
    Which means that while the core functionalities are published as the "COMMUNITY EDITION" with a permissive source-available License (Apache License 2.0) \footcite{PachydermLICENSEMaster}.
    Functionality like \ac{SSO} or the ability to create more than 16 pipelines are part of a different distribution under a Commercial License.

    But in our case this is of no concern, as the startup behind the Pachyderm softwarem, including its \ac{IP} was aquired by \ac{HPE}.
    Giving us a free hand to modify without needing to worry.

    \item \textbf{Argo} is licensed under the Apache License 2.0. \footcite{ArgocdLICENSEMaster}
    Apache 2.0 is quite a standard license permissive license for this kind of product, as it enables  [ TO DO ]
    \item \textbf{\ac{CLASP}} is not a published software and therefore not under any specific license.
    But similar considerations as the ones of Pachyderm apply here aswell, as it is an internal project the \ac{IP} also completely belongs to \ac{HPE}

    \item \textbf{Snaplogic} is an entirely commercial product which does not provide insight into nor the right to modify their Software \footcite{SnapLogicMasterSubscription}.
    But as they might agree this is not a total knockout criterion for this entire project, but in regards to the licensing it will be weighted with 0.
    \item \textbf{Airflow} is licensed under the Apache License 2.0. \footcite{LicenseAirflowDocumentation}
    \item \textbf{Kubeflow} is licensed under the Apache License 2.0. \footcite{KubeflowLICENSEMaster}
    \item \textbf{Knative} is licensed under the Apache License 2.0. \footcite{KnativeDocsLICENSE}
    \item \textbf{Luigi} is licensed under the Apache License 2.0. \footcite{LuigiLICENSEMaster}
    \item \textbf{CWL} is licensed under the Apache License 2.0. \footcite{CwlutilsLICENSEMain}
    
\end{itemize}

\anhangteil{Strategic alignment}

\anhangteil{Ease of Use}

\anhangteil{Maturity}

\anhangteil{Cost}



\begin{table}[htb]
    \centering
    \begin{tabular}{|l|l|l|l|l|} \hline
        \textbf{Criteria}                                          & \textbf{Pachyderm}    & \textbf{Argo}         & \textbf{\ac{CLASP}}   & \textbf{Snaplogic}     \\ \hline
        Ease of use                                                & TBD                   & TBD                   & TBD                   & TBD                    \\ \hline
        Extensibility                                              & TBD                   & TBD                   & TBD                   & TBD                    \\ \hline
        Community, Support \& Docs                                 & TBD                   & TBD                   & TBD                   & TBD                    \\ \hline
        Maturity                                                   & TBD                   & TBD                   & TBD                   & TBD                    \\ \hline
        Strategic alignment                                        & TBD                   & TBD                   & TBD                   & TBD                    \\ \hline
        License                                                    & 10                    & 5                     & 10                    & 0                      \\ \hline
        Cost                                                       & TBD                   & TBD                   & TBD                   & TBD                    \\ \hline

    \end{tabular}
    \caption{Evaluation of the suggested tools}
    \label{tab:evaluation_of_the_suggested_tools}
\end{table}


\begin{table}[htb]
    \centering
    \begin{tabular}{|l|l|l|l|l|l|} \hline
        \textbf{Criteria}                                          & \textbf{Airflow}      & \textbf{Kubeflow}     & \textbf{Knative}      & \textbf{Luigi}        & \textbf{CWL}          \\ \hline
        Ease of use                                                & TBD                   & TBD                   & TBD                   & TBD                   & TBD                   \\ \hline
        Extensibility                                              & TBD                   & TBD                   & TBD                   & TBD                   & TBD                   \\ \hline
        Community, Support \& Docs                                 & TBD                   & TBD                   & TBD                   & TBD                   & TBD                   \\ \hline
        Maturity                                                   & TBD                   & TBD                   & TBD                   & TBD                   & TBD                   \\ \hline
        Strategic alignment                                        & TBD                   & TBD                   & TBD                   & TBD                   & TBD                   \\ \hline
        License                                                    & 5                     & 5                     & 5                     & 5                     & 5                     \\ \hline
        Cost                                                       & TBD                   & TBD                   & TBD                   & TBD                   & TBD                   \\ \hline

    \end{tabular}
    \caption{Evaluation of the additional tools}
    \label{tab:evaluation_of_the_additional_tools}
\end{table}

%% ============================== Diagrams =====================================

\newpage
\anhang{Diagrams}

\anhangteil{Pipeline Communication Swim Lane Diagram}
\label{appendix:pipeline_communication_sld}

\begin{figure}[htb]
  \centering
  \includegraphics[width=17cm]{graphics/pipeline_communication_sld.png}
  \caption[Swimlane Diagram of the communication between the user and Pachyderm]{Swimlane Diagram of the communication between the user and Pachyderm}
  \label{abb:pipeline_communication_sld}
\end{figure}
\footnotetext{Taken from: \cite{IntroPipelines2023}}

\newpage

%% ================================ Code =======================================


\anhang{Minikube installation instructions}
\label{appendix:minikube_installation_instructions}
\lstinputlisting{../quellen/minikube_installation_instructions.md}

\newpage
\anhang{Kubernetes setup scripts}

\anhangteil{Ansible setup script}
\label{appendix:ansible_setup_script}
\lstinputlisting{../../Project/Kubernetes_Setup/kluster/setup_scripts/join_cluster.yaml}

\newpage
\anhangteil{Flannel configuration}
\label{appendix:flannel_config}
\lstinputlisting{../../Project/Kubernetes_Setup/kube-flannel.yml}

% \anhangteil{Bash setup script}
% \label{appendix:bash_setup_script}
% \lstinputlisting{../../Project/Kubernetes_Setup/kluster/setup_scripts/setup.sh}

\newpage
\anhangteil{Bash verification script}
\label{appendix:bash_verification_script}
\lstinputlisting{../../Project/Kubernetes_Setup/kluster/setup_scripts/double_check.sh}

\newpage
\anhangteil{Arkouda Setup}
\label{appendix:arkouda_setup}
\lstinputlisting{../../Project/Kymera/README.md}