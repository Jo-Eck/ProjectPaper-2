\chapter{Creation of the Artifact}
\label{creation_of_the_artifact}

\section{Initial Goals}
\label{artefact_inital_goals}

As this project was first and foremost a project, designed to interactively explore the problem space from the perspective of the \ac{HPC} community, 
all the while being contained by business requirements and time constraints, the initial goals of this project were very broad and open-ended. 
At first the initial goal was simply to create a \ac{PoC} of a realistic workflow engine using the "Arkouda" project,
in order to present the Customer with an easily graspable example of its capabilities.

While we are approaching the problem from the perspective of the \ac{HPC} community, the intended end user of this tool are the data scientists and \acp{SME}
that are working with the \ac{HPC} systems, and therefore the tool needs to be designed and selected with the fact in mind that the end user will most likely not be knowledgeable in the field of \ac{HPC} or the underlying infrastructure.

In the first iteration of the project a preselection of possible Workflow management tools was given from the business side,
with the option to increase the scope if the presented tools were not sufficient.

Therefore, the goals of the first iteration of this project was twofold, first to determine which, if any, of the presented tools were suitable for the task at hand,
and to determine what would make an adequate \ac{PoC} for the customer.

The following iterations are split into the tree main aspects of the project and will be discussed in their own subsections.
While these steps where happening concurrently, they each address a different aspect of the project and therefore underwent their own iterative processes.


\section{Overall Structure}
 
\begin{figure}[htb]
    \centering
    \includegraphics[width=16cm]{graphics/pachykouda_three_aspects.png}
    \caption[Pachykouda high level diagram showing three main aspects]{Pachykouda high level infrastructure diagram}
    \label{abb:pachykouda_three_aspects}
\end{figure}


As can be seen in figure \ref{abb:pachykouda_three_aspects}, the artifact is composed of 3 main components, 
the \textbf{Central Workflow Engine} which is responsible for the orchestration of the workflows (center) and interfaces directly with the underlying infrastructure,
the \textbf{\ac{HPC} Framework} which is responsible for the execution of \ac{TCP} workloads (left)
and the \textbf{Supplementary Services} which aim at improving the usability and accessibility for the end user (right).

All this is build on top of a hardware-agnostic \ac{k8s} cluster, which is responsible for the orchestration of the different components and the underlying infrastructure.


\section{Selection of Workflow Management Tools}

As described in section \ref{artefact_inital_goals}, the first iteration of this project was to determine which, if any, of the presented tools were suitable for the task at hand.
The following section will describe the process of selecting the tools and the criteria that were used to evaluate them.
Because the time frame does not allow for a full integration and testing of all the presented tools in depth we will be using a decision-making framework to evaluate the tools,
as described in the Methodologies \Ref{decision_making} to determine which tools will be most suitable for an initial \ac{PoC} and will serve as a good starting point for the project and future iterations.

\begin{itemize}
    \item \textbf{Pachyderm:} A \ac{k8s} based Workflow manager, written in go which was recently acquired by \ac{HPE}.
    \item \textbf{Argo:} A \ac{k8s} based Workflow manager, written in go, which is a \ac{CNCF} project \footcite{ArgoprojArgoworkflows2023}.
    \item \textbf{\ac{CLASP}:}  An in-house developed workflow manager, written in Java, utilizing Serverlet to execute workflows\footcite{sayersCloudApplicationServices2015}.
    \item \textbf{Snaplogic:} A commercial low-code/no-code workflow manager with a focus on data integration and data engineering\footcite{IPaaSSolutionEnterprise}.
\end{itemize}

But given that it was possible to select projects outside the initial selection, the following projects also need to be considered:


\begin{itemize}
    \item \textbf{Airflow:} A Python-based workflow manager under the \ac{CNCF} umbrella, known for its easy-to-use interface and extensibility\footcite{hainesWorkflowOrchestrationApache2022}.
    \item \textbf{Kubeflow:} A \ac{k8s}-native platform for deploying, monitoring, and running ML workflows and experiments, also a \ac{CNCF} project, streamlining \ac{ML} operations alongside other Kubernetes resources \footcite{Kubeflow}.
    \item \textbf{Knative:} An open-source \ac{k8s}-based platform to build, deploy, and manage modern serverless workloads, simplifying the process of building cloud-native applications \footcite{HomeKnative}.
    \item \textbf{Luigi:} An open-source Python module created by Spotify to build complex pipelines of batch jobs, handling dependency resolution, workflow management, and visualization seamlessly \footcite{SpotifyLuigi2023}.
    \item \textbf{\ac{CWL}:} An open-standard for describing analysis workflows and tools in a way that makes them portable and scalable across a variety of software and hardware environments, from workstations to cluster, cloud, and high-performance computing environments.
\end{itemize}
    
\subsubsection{Selection Criteria}

Due to this extensive list of diverse tools, a set of criteria was established to determine which tool would be the most suitable for the task at hand.
The following list of criteria was established to evaluate the tools:

\begin{itemize}
    \item \textbf{Ease of use:} 
        As the hinted end users of the tool are not primarily \ac{HPC} experts, the tool needs to be easy to use and understand,
        and should not require the end user to have a deep understanding of the underlying infrastructure.
        While we can expect that the administration of the infrastructure will be done by adequately trained personnel, 
        the end users should be spared having to adapt to the underlying infrastructure as much as possible.

    \item \textbf{Extensibility:}
        One significant constraint of the project is the restricted number of available work-hours.
        Given that the project's environment predominantly centers around HPC (High Performance Computing) workloads,
        it's essential for the tool to be easily expandable without requiring extensive modifications to the underlying system.
        Ideally this property would be transferred to the end users, allowing them to easily extend the developed tool further to their needs.

    \item \textbf{Community, Support and  Documentation:}
        It is not enough that the software technically permits extensibility, the software also needs to be adequately documented, and a support framework needs to be in place.
        Be it a community of users or a dedicated support team, the end users and the developers need to be able to rely on the software being maintained and updated as well as being able to find expert help in case of problems.

    \item \textbf{Maturity:}
        With the boom of \ac{AI} and \ac{ML} in recent years \footcite{24TopAI}, the number of tools and frameworks has exploded, and while this is a good thing it also means that a lot of these tools are still paving their way and are developing rapidly.
        While this is not necessarily a bad thing, it does mean that the tool might not be ready for production use and might not be able to provide the stability and reliability that is required for a production environment or are lacking in documentation and support.      

    \item \textbf{Strategic alignment with \ac{HPE}:}
        As this project is being developed within the context of \ac{HPE}, it is important to consider the strategic alignment of the tool with \ac{HPE}.
        \ac{HPE} has is a large company with a diverse portfolio of products and services, and this project intersects with many parts of the company.
        Therefore, it is important to consider the strategic alignment of the tool with \ac{HPE} and its products and services.

    \item \textbf{License:}
        \label{crit:license}
        While this \ac{PoC} is not a commercial product in itself but rather an exploration of the problem space and a demonstration of what a final commercial product  might be like,
        it is important to consider the licenses of the tools that are being used.
        Having to strip out a tool later on because of licensing issues would be a significant setback and therefore needs to be considered.

    \item \textbf{Cost:}
        Time is not the only constraint of this project, as the project is being developed within the context of \ac{HPE} it is important to consider the cost of the tools that are being used.

\end{itemize}

\subsubsection{Weighing of the Criteria}

An integral part of the \ac{SMART} methodology is the weighting of the criteria, as described in section \ref{decision_making}.
In order to rank the criteria themselves, as they are quite hard to quantify, 
We will be using the weighing methodology as described in the \ac{SMART-ER} methodology \ref{acro:SMART-ER}.

The first step of which is the ranking of the criteria from most important to least important.

\begin{enumerate}
    \item \textbf{ Extensibility } As this is first and foremost a prototyping project, the actual development it at least for the first couple steps of the highest importance. 
    \item \textbf{ Community, Support \& Docs } This also applies for the external support available to the development team as if they are stuck, no developed can proceed, no matter the other factors.
    \item \textbf{ License } This criterion has to weighted carefully, as a highly restrictive license might be a deal-breaker, but a license that is too permissive might conflict with the strategic alignment with \ac{HPE}.
    \item \textbf{ Strategic alignment with \ac{HPE} } As this is developed by and for \ac{HPE} their requirements need to be considered as well.
    \item \textbf{ Ease of Use } While the ease of use is important as this should eventually become a product, for now the central aspect is to create a \ac{PoC} therefore the usability is a priority, but not the highest.
    \item \textbf{ Cost } As this is a \ac{PoC} and not a commercial product, the cost is not the highest priority as this will be of small scale and therefore the cost will be negligible in most cases.
    \item \textbf{ Maturity } While the maturity of the tool is important, as this is a \ac{PoC} and not a commercial product, if the maturity of the tool does not impact the extensibility of the tool or the development process, it is not the highest priority.
\end{enumerate}

As all these criteria are quite important, the weighting function selected for the criteria is the \ac{RS} function, as described in section \ref{smart_er}, 
as it does not rank the criteria too harshly.
The lookup tables for the weighting function can be found in the appendix \ref{abb:pipeline_communication_sld}.

\begin{table}[htb]
    \centering
    \begin{tabular}{|l|l|} \hline
        \textbf{Criteria}                       & \textbf{Weight}       \\ \hline
        Extensibility                           &  0.2500               \\ \hline
        Community, Support and  Documentation   &  0.2143               \\ \hline
        License                                 &  0.1786               \\ \hline
        Strategic alignment with \ac{HPE}       &  0.1429               \\ \hline
        Ease of use                             &  0.1071               \\ \hline
        Maturity                                &  0.0714               \\ \hline
        Cost                                    &  0.0357               \\ \hline

    \end{tabular}
    \caption{Weighting of the criteria}
    \label{tab:weighting_of_the_criteria}
\end{table}

\subsubsection{Evaluation of the Tools}

Now that we have established the criteria as well as their weighing, we can begin to evaluate the tools based on the criteria.
Here we will be using a mix of Methodologies, as some of these criteria can simply be indexed via analogous values, while others are of a more non-specific nature.
The discussion of which values will be used on which weighing scale for the tools' comparison can be found in the appendix under 

\begin{table}[htb]
    \centering
    \begin{tabular}{|l|l|l|l|l|} \hline
        \textbf{Criteria}                                          & \textbf{Pachyderm}    & \textbf{Argo}         & \textbf{\ac{CLASP}}   & \textbf{Snaplogic}     \\ \hline
        Ease of use                                                & TBD                   & TBD                   & TBD                   & TBD                    \\ \hline
        Extensibility                                              & TBD                   & TBD                   & TBD                   & TBD                    \\ \hline
        Community, Support \& Docs                                 & 8.43                   & 2.32                   & 2.5                   & 5.03                   \\ \hline
        Maturity                                                   & TBD                   & TBD                   & TBD                   & TBD                    \\ \hline
        Strategic alignment                                        & 10                   & 2.5                   & 7.5                   & 0                    \\ \hline
        License                                                    & 10                    & 7.5                     & 10                    & 0                      \\ \hline
        Cost                                                       & TBD                   & TBD                   & TBD                   & TBD                    \\ \hline

    \end{tabular}
    \caption{Evaluation of the suggested tools}
    \label{tab:evaluation_of_the_suggested_tools}
\end{table}

The following table shows the evaluation of the tools which where chosen for their relevance to the problem space, based on the criteria and the weighting of the criteria:

\begin{table}[htb]
    \centering
    \begin{tabular}{|l|l|l|l|l|l|} \hline
        \textbf{Criteria}                                          & \textbf{Airflow}      & \textbf{Kubeflow}     & \textbf{Knative}      & \textbf{Luigi}        & \textbf{CWL}          \\ \hline
        Ease of use                                                & TBD                   & TBD                   & TBD                   & TBD                   & TBD                   \\ \hline
        Extensibility                                              & TBD                   & TBD                   & TBD                   & TBD                   & TBD                   \\ \hline
        Community, Support \& Docs                                 & 10                    & 2.25                  & 0.74                  & 2.29                  & 0.22                  \\ \hline
        Maturity                                                   & TBD                   & TBD                   & TBD                   & TBD                   & TBD                   \\ \hline
        Strategic alignment                                        & 2.5                   & 2.5                   & 2.5                   & 2.5                   & 2.5                   \\ \hline
        License                                                    & 7.5                   & 7.5                   & 7.5                   & 7.5                   & 7.5                   \\ \hline
        Cost                                                       & TBD                   & TBD                   & TBD                   & TBD                   & TBD                   \\ \hline

    \end{tabular}
    \caption{Evaluation of the additional tools}
    \label{tab:evaluation_of_the_additional_tools}
\end{table}

\subsubsection{Conclusion of the Selection Process}

\textcolor{red}{TODO: Write conclusion of the selection process}

\blindtext[1]


\section{Implementation of the Artifact}

This section will describe the iterative process of implementing the larger artifact and is broken up into 3 subsections.
While these steps where happening concurrently, they each address a different aspect of the project and therefore mostly underwent their own iterative processes.



\subsection{Infrastructure}


\subsubsection{First iteration - Minikube}
\label{minikube}
As the decision of the Workflow management tool was made, it was obvious that a dedicated \ac{k8s} infrastructure was needed to run the tool\footcite{PachydermDocsOnPrem}.
The Pachyderm documentation gave two recommendations for setting up an initial development environment, preferably Docker Desktop or alternatively Minikube \footcite{PachydermDocsLocal}.
Due to the exclusive license of Docker-Desktop\footcite{DockerTermsService2022},
which prevents large companies free usage of the product\footcite{DockerFAQsDocker2021} the choice fell on Minikube for an initial test setup.

In addition to the underlying \ac{k8s} Pachyderm also needs an external S3 Storage Bucket for its \ac{PFS} for which we used MinIO,
a self-hostable S3 compliant object storage\footcite{incMinIOMinIOKubernetes}, which was also based on recommendations by the Pachyderm documentation.

The persistent storage requirements for the Pachyderm itself was fulfilled by manually creating two \ac{PV}'s on the hosts local hard drive.
Using the Helm packagemanager\footcite{HelmDocsHome} for \ac{k8s} the at that point, the newest version 2.6.4, was installed from the official Artifacthub repository\footcite{ArtifacthubPachyderm}.

The host system of this iteration was a single ProLiant DL385 Gen10 Plus running Ubuntu 22.04.3 LTS x86\_64.
During the setup every step was diligently noted and put into a repository\footcite{eckerthInstallationInstructionsMinikube}, alongside the needed scripts. 
The instructions can be found in the appendix at \ref{appendix:minikube_installation_instructions}.


\subsubsection*{Learnings from the first iteration}

The shortcomings of this naive first iteration became apparent very quickly, 
which was to be expected, as the goal of this iteration was to create a minimal working example to get a better understanding of the tooling and the underlying infrastructure.

The first and foremost issue were the limitations imposed by Minikubes' reliance on an Internal \ac{VM}.
During testing the inability to increase the resources of the \ac{VM}  on the fly  became a significant bottleneck.
At some point during the testing of \ref{tcp_hpc_workloads} the \ac{VM} was so overloaded that the installation was irreparably damaged which was seen as a sign to move on to the next iteration.

Another more subtle issue was the discrepancy between the experience a small scale \ac{k8s} installation within Minikube and a large scale \ac{k8s} cluster like the one that would be used in later steps of the project.
Therefore, it was decided that a more realistic \ac{k8s} cluster would be needed for the next iteration, which became the Heydar cluster.

\subsubsection{Second iteration - Heydar Cluster}
\label{heydar_cluster}

Improving upon the shortcomings of the first iteration, the second iteration was based in the attempt to create a more realistic \ac{k8s} cluster.
To achieve this, 20 ProLiant DL360 Gen9 Servers, running Ubuntu 22.04.3 LTS x86\_64 were used to create a bare metal \ac{k8s} cluster,
using kubeadm as it provides deep integration with the underlying infrastructure\footcite{CreatingClusterKubeadm}.

However a bare metal cluster also comes with its own set of challenges, as the cluster needs to be provisioned and configured manually.
In order to automate this process, the Ansible automation tool was used to set up all the nodes in parallel and to ensure that the all the nodes are in the same state.
Ansible is a declarative tool which allows for the automation of the provisioning and configuration of the cluster\footcite{Ansible2023}, by specifying the desired state of the cluster in a playbook and then applying it to the cluster.
The Ansible playbook used for the setup of the cluster can be found in the projects repo\footcite{eckerthProjectRepoAnsible}.

The application of this configuration on the cluster unknowingly caused conflict between the Ansible playbook and the maintenance scripts of the cluster as the Heydar machines.
As \ac{k8s} needs very specific configurations on the underlying infrastructure like the deactivation of swap space\footcite{InstallingKubeadm}.

This was resolved by consulting with the maintainer of the cluster and adjusting the Ansible playbook as well as the maintenance config for the cluster nodes accordingly, 
after we had identified the issue.


One important aspect of a production like cluster is the networking, as \ac{k8s} does not natively manage communication on a cluster level,
but instead relies on so called \ac{CNI}s to manage and abstract the underlying network infrastructure \footcite{ClusterNetworking}.

Here we are spoiled for choice once again, as there are a multitude of different \ac{CNI}s available, each with their own advantages and disadvantages.
The Kubernetes documentation provides a non-exhaustive list of 17 different \ac{CNI}s\footcite{KubernetesCNIPlugins}, which all fulfill this essential task in different ways.
As the needs regarding the network plugin were not very specific at this point, the choice fell on Calico, as surface level research showed that it was a popular choice for bare metal clusters\footcite{ExploreNetworkPlugins},
provided security and enterprise support, as well having a wide range of features\footcite{mehndirattaComparingKubernetesContainer}.
However Calico proved to be more difficult to set up than expected, after consulting with a college who set up a different cluster with Calico,
it was decided to use Flannel as a \ac{CNI} instead.
Flannel turned out to be much easier to set up and configure, as it is a very lightweight \ac{CNI} which is designed for bare metal clusters\footcite{Flannel2023}, 
and foregoes the more advanced security features of Calico. 

The Flannel configuration used for the cluster can be found in the project repo\footcite{eckerthProjectRepoFlannel}, it is closely based on the example configuration provided by the Flannel documentation\footcite{FlannelInstallConfig}.

\subsubsection*{Learnings from the second iteration}

The second iteration was a significant improvement over the first iteration, as it provided a much more realistic environment for the development of the artifact.
This also came with its own set of challenges, as the bare metal cluster needed to be provisioned and configured manually, which was a significant time investment.

What became apparent very quickly was that the solution for the provisioning of the \ac{PV} was nowhere near scalable,
as it relies on the local hard drive of the host machine and therefore must host the container on the same machine as the \ac{PV} which defeats the purpose of a multi node cluster in the first place.
Therefore, a more scalable solution needs to be implemented for the next iteration.
A possible solution could be the use of distributed storage solutions like Ceph\footcite{CephIoHome} or GlusterFS\footcite{Gluster}  in combination with the Rook project \footcite{Rook},
which will need to be explored in future iterations.


As described in section \ref{third_iteration_fam} a service hosting \ac{FAM} will be needed in future iterations as well.



\subsection{Tightly Coupled HPC Workloads} 
\label{tcp_hpc_workloads}

As described in section \ref{state_of_the_art_tcp} \ac{TCP} problems are a large part of the \ac{HPC} world,  but seem to lack native support in Pachyderm.
Pachyderm as it exists as of writing this thesis, is centralized around \ac{LCP} problems, as it is designed to work with large amounts of data but with each so called "datum" being independent of each other.
This is a very good fit for \ac{LCP} problems, and ties into their concepts of data lineage, versioning and providence.

\begin{figure}[htb]
    \centering
    \includegraphics[width=14cm]{graphics/datum_distribution_amongst workers.png}
    \caption[Pachykouda datum distribution amongst workers]{Pachykouda datum distribution amongst workers \footnotemark}
    \label{abb:datum_distribution_amongst workers}
\end{figure}
\footnotetext{Taken from: \cite{IntroPipelines2023}}


Diagram \ref{abb:datum_distribution_amongst workers} shows Pachyderms approach to distribute their datums amongst workers, given an already defined pipeline.
Once Data files are added to the input repository, Pachyderm will determine Based on a glob pattern wether the files are relevant datums for the pipeline.
If the newly added data fits the pattern each of the files will be supplied to its own instantiation of a worker, all originating from the same image, which will then process the data concurrently and independently of each other.
After the worker has finished its task, the resulting datums are then collected in their own repository of data.
A more detailed swim lane diagram of this process can be found in the appendix at \ref{abb:pipeline_communication_sld}

This approach is very well suited for \ac{LCP} problems, as the datums are independent of each other and can be processed in parallel without any issues.
But it is not well suited for Large \ac{TCP} problems, if the computation of the data can not be split into distinct independent datum files, or the computation is reliant on the intercommunication of the datums.
If the datasets are small enough, this does not really present a problem as one can simply take all the data into a single worker node and process it there.
But as a single worker node can only utilize the resources of a single physical compute node, this does not scale well with the size of the dataset and defeats the purpose of a distributed system in the first place.

So our goal for this section is a way to find a way to enable pachyderm to pool the entire resources of the cluster, in order to solve a \ac{TCP} problem.

\subsubsection{First iteration - PachyKouda}

As a first attempt to address this issue, it was decided that the integration of a \ac{TCP} framework into Pachyderm on the container level would be the best approach.
So the first iteration is based on the idea of a Pachyderm conforming client container, which is able to interface with an external \ac{TCP} framework,
which can handle the reception of the data, the distribution of the data amongst the workers and the collection of the results to reintegrate them into the \ac{PFS}.

The first iteration of this idea was called PachyKouda, as it was based on the Arkouda \ac{TCP} framework\footcite{ArkoudaGituhbRepository2023},
which itself is a python binding for the Chapel programming language \footcite{ChapellangChapelProductive}. 

For that step an Arkouda worker was installed bare metal on the head node of the Heydar cluster, in order to verify the feasibility of the idea,
with the goal of moving the worker into the cluster in the next iteration.

The client container was based on the official \ac{UDP}-based build by the Arkouda team \footcite{ArkoudacontribArkoudadockerMain}.
The container was then modified to be able to communicate with the Arkouda worker on the head node of the cluster, it can now send data to the worker and receive the results.

\subsubsection*{Learnings from the first iteration}

The first iteration was a total success, as it proved the feasibility of being able to use a client container to forward the data processing to an external Arkouda worker.
As described earlier, the goal of the next iteration is to move the Arkouda worker into the cluster, in order to be able to utilize the full resources of the cluster.

\subsubsection{Second iteration - Kymera}

\begin{figure}[htb]
    \centering
    \includegraphics[width=16cm]{graphics/PachyKouda.png}
    \caption[Arkouda workers on \ac{k8s}]{Arkouda workers on the Heydar cluster}
    \label{abb:arkouda_workers_on_k8s}
\end{figure}

Diagram \ref{abb:arkouda_workers_on_k8s} above shows a high level overview of how the workers interface with the client container in the workflow.
The Arkouda container which is part of the workflow is still the same as in the first iteration, but now instead of interfacing with an external worker it 
is interfacing with a worker swarm hosted across the cluster.

The Swarm is split into two parts, one central Arkouda server, facilitating the communication between the client container and the workers and the workers also called locales themselves.
The locales and the server are based on the helm charts provided by the Arkouda-Contrip repo\footcite{BearsRUsArkoudacontribArkoudahelmcharts},

A detailed walk through the setup of the \ac{RBAC}, Secrets and deployments for the Heydar Cluster can be found in the appendix at \ref{appendix:arkouda_setup} which in turn is based on the official
Arkouda documentation\footcite{ArkoudacontribArkoudadockerMain}.

\subsubsection*{Learnings from the second iteration}
As Arkouda does not currently provide multi tenancy of their Server, meaning that they can only be connected a single client at a time, 
so if multiple pipelines need to solve a \ac{TCP} at the same time, they would not be able to share the same worker swarm.
Instead, they would need to spawn their own worker swarm.

Another issue is that there are currently going through the standard Pod to pod communication configuration of flannel, which means that the entire traffic between the client container and the Arkouda server 
as well as the traffic between the workers is all happening over emulated overlay network which enables the containers on the different nodes to communicate with each other as if they where on the same network, no matter of the actually infrastructure below it.
The communication protocol of the Arkouda servers is \ac{UDP} based \ac{GASNet}, which provides the \ac{RDMA} needed for the Arkouda framework to work, but this incurs a significant overhead in the form of the encapsulation of the \ac{UDP} packets into \ac{TCP} packets.

Also, the containers are currently not compatible with the OpenFAM project \footcite{keetonOpenFAMAPIProgramming2019}, which is being developed as an integration to Arkouda and Chapel by the Hewlett Packard Systems Architecture Lab \footcite{byrneCouplingChapelPoweredHPC2023},
it extends the Arkouda framework with the ability to use \ac{FAM} as banks of \ac{RDMA} enabled memory, which can be accessed by the Arkouda workers.
This would proof to be a significant improvement as it has the potential to reduce the overall overhead of the communication\Footcite{chouOptimizingPostCopyLive2019} amongst the workers as well as to the server, by cutting down the overall amount of network traffic.

The pachyderm platform itself might also benefit from the integration of \ac{FAM}, as it could be used to store the datums in the \ac{PFS}, providing the running pipeline processes with a much faster access to the data.

\section*{Third iteration - \ac{FAM}}
\label{third_iteration_fam}
While significant efforts have already been made to successfully integrate Arkouda and \ac{FAM}, these have so far been focussing on bare metal installations, for that reason, in order to integrate the \ac{FAM} enabled Arkouda working from within a containerized environment the tools would need to be 
custom recompiled matching the new environment. 
Therefore, we needed to:

\begin{enumerate}
    \item Compile OpenFAM in the Container
    \item Compile custom Chapel in the Container with OpenFAM
    \item Compile custom Arkouda in the Container with the OpenFAM enabled Chapel
    \item Rebuild the Arkouda container with the new Arkouda binary
    \item Reweite the \ac{k8s} deployment to make use of OpenFAM
\end{enumerate}

This section was quite challenging as it required a deep understanding of the \ac{PoC} implementations of the OpenFAM, Arkouda and Chapel projects and was cut short by the time constraints of the project and was therefore not brought to a successful conclusion.
\textcolor{red}{The current state of the project can be found in the \ac{PoC} repository\footcite{eckerthPoCRepository2023} and in the appendix at \ref{appendix:arkouda_fam}. 
}
But this showed us that there is a lot of potential in the integration of \ac{FAM} into the container based \ac{HPC} world, as it could provide a significant performance boost to the overall system and should be explored further in future iterations.


\subsection{Supplementary Services}

As the other branches of the prototyping where happening, the need for additional services and infrastructure arose to 
support the development of the prototype as well as to increase the general usability of the prototype. 
This section will especially describe the services which help to make this prototype a more complete solution.


\begin{figure}[htb]
    \centering
    \includegraphics[width=17cm]{graphics/pachykouda_complete.png}
    \caption[Pachyderm High-Level Architecture]{Pachyderm High-Level Architecture}
    \label{abb:pachyderm_complete}
\end{figure}

\subsubsection{Docker Registry}

One thing that was quite apparent from the getgo, was the need for a central docker registry.
As Pachyderm does not manage the docker images itself, but relies on the user to provide them somehow externally.

During the first iterations when the development was being done on Minikube as described in \ref{minikube}, the internal Registry 
of the node was enough.
But as soon as we moved over to the Heydar system keeping the Hosts internal registries in sync was of course not feasible,
Therefore we added a private docker registry to the cluster \footcite{kumarHowSetupPrivate2020}.

\subsubsection{Frogejo Catalogue}
\subsubsection{Jenkins CI/CD Pipeline}





\section{Evaluation of the Artifact}


The original problems described in \ref{ProblemStatement} where seven-fold, and where addressed in the following way:

\begin{itemize}
    \item \textbf{Workload Resilience and Fault Tolerance in HPC:}
    A problem which is typically addressed by the \ac{HPC} community by using a combination of checkpointing and job scheduling\footcite{jinOptimizingHPCFaultTolerant2010}
    Is now being directly address via the inclusion of pachyderm.
    Because Pachyderm isolates each of the processes into their own container, and tracks each of the steps induvidually
    it can easily restart a failed step, or even a failed job, without having to restart the entire workflow.
    While this only works for the standard \ac{LCP} workloads, its provenance features reduce the data loss, should a \ac{TCP} workload fail.
    \item \textbf{Environment/Package Management in HPC:} 
    Replacing the classical \ac{HPC} package management solutions with a containerized approach,
    simplifies the deployment of code from the users' perspective massively, as they have almost complete control of the environment their code 
    is going to run in, all the while giving the administrators the ease of mind that the code is not going to interfere with the rest of the system.
    \item \textbf{Probability issues with HPC:}
    Same goes for the probability issues, as the containerized approach allows for a much more fine-grained control of the environment,
    users can rapidly iterate and test their code on their local machines, before deploying it to the \ac{HPC} system.
    \item \textbf{Scalability issues with HPC:}
    Scalability is one of the main features of \ac{k8s}, and therefore of Pachyderm a cluster can easily be scaled up or down, depending on the current workload.
    Many cloud providers even offer hosted \ac{k8s} clusters, which can be scaled up or down on demand, and therefore allow for a very flexible approach to the problem.
    While in classical \ac{HPC} systems, the cluster is usually fixed in size, and therefore the user has to wait for the next available slot.
    \item \textbf{Interconnected Problem-Solving in \ac{CC}}
    This one was one of the problems which was not directly address by pachyderm or Kubernetes directly,
    in order to solve this problem, Arkouda was containerized and made usable for Pachyderm workloads.
    As of right now, the layers of network abstractions and the lack of OpenFAM support have a negative impact on the performance of the \ac{TCP} workloads,
    but successfully proofs the concept of interconnected problem-solving on a \ac{CC} system.
    \item \textbf{Provenance and Versioning:}
    Combining the advantages the Pachyderm File System with the completely \ac{CI/CD} based approach to the deployment of the workflows,
    allows a tracking of each and every part that goes into each and every step of the workflows.
\end{itemize}


While this project does not present a complete solution to all the problems, it does present a viable path forward for a more modern approach to \ac{HPC}.
The combination of \ac{HPC} and Pachyderm allows for a much more flexible approach to the problem and with future work and low level driver support and usability 
features like the Jenkins Pipeline, as well as a well maintained ecosystem of pipeline steps which can be used to build more complex workflows and reutilize existing code
developed by fellow researchers, this approach could be a significant improvement over the current state of the art.

Unfortunately the time was cut short before the project could be fully completed and therefore some goals like the integration of OpenFAM, the switch to a low abstraction \ac{CNI} and especially 
multi parameter performance testing could not be completed in time, it is apparent that the integration of these would bring the project much closer to the performance  of the classical \ac{HPC} systems,
while still maintaining the flexibility and ease of use of the containerized approach, and therefore should be considered for future work.
