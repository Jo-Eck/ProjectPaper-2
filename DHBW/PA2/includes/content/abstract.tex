\begin{abstract}
\thispagestyle{kapitelkopfzeile}
\textbf{Abstract:}

This Second project paper presents the development and demonstration of a proof-of-concept for the 
integration of \ac{HPC} programming frameworks inta a container-base workflow orchestrator the convergence of
\ac{HPC} and \ac{CC} has revealed novel potential in highly scalable and flexible computing.

This project aim to reconciel the different demands of the \ac{HPC} and \ac{CC} communities by demonstating the integration of 
the \ac{HPC} programming framework Arkouda into the container-based workflow orchestrator Pachyderm, showing the technical feasibility of this approach.

A prototype implementing this integrated system is constructed and evaluated through prototyping methodologies, with a focus on resilience, scalability, portability, and user-friendliness.
The prototype is iteratively refined to address \ac{LCP} and \ac{TCP}, with particular attention to the usability of the system for non \ac{CC} experts.

This project paper contributes to the body of knowledge by way of practical example, lessons learned with each iteration, and sheds lith on pathways for future research towards
a landscape where the seamless and efficient integration of \ac{HPC} workloads in \ac{CC} environments becomes possible

\end{abstract}